\documentclass[10pt, letterpaper]{developercv}
\usepackage{enumitem}
\usepackage{setspace}

\begin{document}

\setlist{nosep}
\setlist{leftmargin=1.5em}
\setstretch{1.1}
\raggedright


%----------------------------------------------------------------------------------------
% Title and Contact Info
%----------------------------------------------------------------------------------------

\begin{cvminipage}{0.35}
    \HUGE{\MakeUppercase{\textbf{Nick\\Stevens}}}
\end{cvminipage}
\begin{cvminipage}{0.28}
    \icon{MapMarker}{12}{Brooklyn Center, MN}\\
    \icon{Phone}{12}{(651) 402-1610}
\end{cvminipage}
\begin{cvminipage}{0.35}
    \icon{At}{12}{\href{mailto:nickastevens83@gmail.com}{nickastevens83@gmail.com}}\\
    \icon{Github}{12}{\href{https://github.com/nastevens}{github.com/nastevens}}
\end{cvminipage}
\vspace{3ex}


%----------------------------------------------------------------------------------------
% Introduction and Skills
%----------------------------------------------------------------------------------------

\begin{cvminipage}{0.5}
    \cvsect{Introduction}
    Experienced software engineer with deep knowledge of network-connected
    embedded devices and associated back-end cloud services. Linux expert.
    Knowledgable in hardware and PCB design for embedded product development.
\end{cvminipage}
\hfill
\begin{cvminipage}{0.45}
    \cvsect{Skills}
    \begin{cvminipage}{0.3}
        \begin{itemize}[nosep]
            \item Rust
            \item Yocto
            \item Python
            \item C/C++
            \item Shell
        \end{itemize}
    \end{cvminipage}
    \hfill
    \begin{cvminipage}{0.3}
        \begin{itemize}[nosep]
            \item Lua
            \item JavaScript
            \item Terraform
            \item Java
            \item Groovy
        \end{itemize}
    \end{cvminipage}
    \hfill
    \begin{cvminipage}{0.3}
        \begin{itemize}[nosep]
            \item AWS
            \item Android
            \item \LaTeX
            \item Git
        \end{itemize}
    \end{cvminipage}
\end{cvminipage}
\vspace{3ex}


%----------------------------------------------------------------------------------------
% Experience
%----------------------------------------------------------------------------------------

\cvsect{Experience}
\begin{entrylist}
    \entry
        {2016 -- 2023}
        {Samsung SmartThings}
        {Senior Staff Software Engineer}
        {%
            \texttt{Rust}\slashsep
            \texttt{C}\slashsep
            \texttt{Python}\slashsep
            \texttt{Yocto}\slashsep
            \texttt{Shell}\slashsep
            \texttt{Lua}\slashsep
            \texttt{AWS + Terraform}
            \begin{itemize}[nosep]
                \item Designed and implemented secure firmware update server
                    and client for Hub products, written in 100\% Rust. Server
                    was extremely efficient, running at a fraction of the cost
                    of other SmartThings cloud services, while maintaining very
                    high uptime with no unplanned outages in 6 years.
                \item Wrote Yocto recipes for building the Hub firmware
                    operating system, including advanced classes for automating
                    firmware image encryption and signing.
                \item Performed board bring-up for Hub 2018 model, including
                    U-boot and Linux kernel customization work.
                \item Implemented full-disk encryption and hardware-backed
                    secure boot setup for Hubs.
                \item Created a streaming log encryption and compression tool
                    for securely retrieving logs from hubs. Open-sourced the
                    encryption envelope format as Saltlick at
                    \href{https://github.com/saltlick-crypto}{github.com/saltlick-crypto}
                \item Replaced outdated and buggy Hub manufacturing tools with
                    streamlined tool created in Python for Hub 2015 and 2018
                    models.
                \item Overhauled CMake build of primary Hub application to
                    provide easy cross-compilation of mixed C, C++, and Rust
                    codebase.
                \item Wrote numerous Hub system utilities in Rust and Shell.
                \item Stood up containerized infrastructure in AWS Elastic
                    Container Service (ECS) with Jenkins CI to run heavyweight
                    Yocto builds automatically.
                \item Created in-depth technical proposals and documentation
                    for all of the above.
            \end{itemize}
        }
    \entry
        {2014 -- 2016}
        {Digi Wireless Design Services}
        {Senior Software Engineer}
        {%
            \texttt{C}\slashsep
            \texttt{Python}\slashsep
            \texttt{Yocto}
            \begin{itemize}[nosep]
                \item Developed Yocto-based Linux board support package for
                    Freescale ARM7 board.
                \item Wrote Linux kernel RS-485 driver for Freescale i.MX28
                    processor.
                \item Back-ported Linux Bluetooth drivers from kernel 3.17 to
                    3.14.
                \item Submitted Linux kernel bug patch for MCP3021 analog
                    to digital converter (SHA 347d7e45).
                \item Wrote platform-independent implementation of \texttt{expect}
                    library in Python with full support for Unicode and binary
                    data.
            \end{itemize}
        }
    \entry
        {2012 -- 2014}
        {QiG Group (c/o Greatbatch)}
        {Firmware Engineer}
        {%
            \texttt{C}\slashsep
            \texttt{C++}\slashsep
            \texttt{Java}\slashsep
            \texttt{Groovy}
            \begin{itemize}[nosep]
                \item Participated in cross-functional hardware/firmware team
                    working to re-certify an implantable medical device
                    acquired by QiG group from a defunct company. Specifically
                    responsible for a full code audit of the C/C++ firmware
                    code base.
                \item Created proof-of-concept Android application for
                    communicating with custom USB hardware to retrieve data
                    from the implanted device and upload that data to a cloud
                    data collection service (written in Python).
                \item Developed Android application for emulating a limited
                    Hayes command set and Point-to-Point Protocol (PPP)
                    implementation to allow a legacy device to communicate over
                    a cellular modem as if talking to a Bluetooth dial-up
                    modem.
            \end{itemize}
        }
    \entry
        {2012 -- 2013}
        {FPX}
        {Software Engineer (Contractor)}
        {%
            \texttt{C++}\slashsep
            \texttt{Java}
            \begin{itemize}[nosep]
                \item Performed 32-bit to 64-bit conversion of 70,000-line
                    cross-platform (RedHat Linux and Windows) C++ application
                    as part of three-person team.
                \item Rewrote build system based on Borland tools using GNU
                    Make and MinGW.
            \end{itemize}
        }
    \entry
        {2009 -- 2012}
        {ProMetric Systems}
        {Software Engineer}
        {%
            \texttt{LabVIEW}\slashsep\texttt{C}\slashsep\texttt{Java}\slashsep\texttt{Verilog}
            \begin{itemize}[nosep]
                \item Created serial communication wedge using Verilog and a
                    Spartan 3 FPGA to bridge a standard PC USB port to five
                    serial busses (Two \iic{} busses, a One-Wire bus, and two
                    proprietary serial busses).
                \item Worked on an agile team implementing and maintaining
                    LabVIEW-based test platform for simultaneous testing up to
                    450 rechargeable batteries.
                \item Wrote and validated Java library for Yokogawa SL1000 DAQ
                    instrument.
            \end{itemize}
        }
    \entry
        {2006 -- 2009}
        {Boston Scientific Cardiac Rhythm Management}
        {Electrical Engineer, Manufacturing Test}
        {%
            \texttt{LabVIEW}\slashsep\texttt{Java}\slashsep\texttt{Python}
            \begin{itemize}[nosep]
                \item Developed Java desktop application for easily reading and
                    displaying data from pacemaker memory dumps.
                \item Worked with team of developers to create a LabVIEW-based
                    high voltage test system for implantable cardiac leads.
                \item Wrote and distributed Excel macros and Python scripts for
                    processing field return data.
            \end{itemize}
        }
\end{entrylist}

%----------------------------------------------------------------------------------------
% Education
%----------------------------------------------------------------------------------------

\cvsect{Education}
\begin{entrylist}
    \entry
        {2001 -- 2006}
        {Rose-Hulman Institute of Technology}
        {B.S. in Computer Engineering}
        {%
            \texttt{GPA: 3.89}
            \begin{itemize}[nosep]
                \item Minor in Japanese.
                \item Co-op student for Johnson \& Johnson Consumer Products
                    for a total of 12 months.
                \item Teacher's assistant for Digital Design I and II.
            \end{itemize}
        }
\end{entrylist}

\end{document}
